\documentclass[master=ewit,english]{kulemt}
\setup{title={Building a state of the art speech recogniser},
  author={Moritz Wolter},
  promotor={Prof.\,dr.\,ir.\ Patrick Wambacq \and Prof.\,dr.\,ir.\ Hugo van Hamme},
  assessor={Prof.\,dr.\,ir.\ Johan Suykens},
  assistant={Ir. Vincent Renkens}}
% The following \setup may be removed entirely if no filing card is wanted
\setup{filingcard,
  translatedtitle=,
  udc=621.3,
  shortabstract={
  	In the past machine learning relied heavily on algorithms designed by experts to solve a specific task. Which lead to highly sophisticated algorithms, which could be grasped only by small groups of people. The human brain however does not work this way, although specialized areas exist, these areas consist of similar
  	building blocks. Artificial neural networks attempt to mimic this layout. Similar algorithmic structures are used for a wide variety of tasks. This thesis deals with\ the application of neural networks in speech recognition.  Replacing the various	subsystems by one integrated network based approach.   
    \endgraf}}
% Uncomment the next line for generating the cover page
%\setup{coverpageonly}
% Uncomment the next \setup to generate only the first pages (e.g., if you
% are a Word user.
%\setup{frontpagesonly}

% Choose the main text font (e.g., Latin Modern)
\setup{font=lm}

% If you want to include other LaTeX packages, do it here. 
% ---------- Titelblad Masterproef Faculteit Wetenschappen -----------
% Dit document is opgesteld voor compilatie met pdflatex.  Indien je
% wilt compileren met latex naar dvi/ps, dien je de figuren naar
% (e)ps-formaat om te zetten.
%                           -- december 2012
% -------------------------------------------------------------------
\RequirePackage{fix-cm}

% --------------------- In te laden pakketten -----------------------
% Deze kan je eventueel toevoegen aan de pakketten die je al inlaadt
% als je dit titelblad integreert met de rest van thesis.
% -------------------------------------------------------------------
\usepackage{graphicx,xcolor,textpos}

%----------------------- Custom stuff -------------------------------

\graphicspath{./}
\usepackage{makeidx}
\usepackage{amsmath}
\usepackage{amssymb}
\usepackage[english]{babel}
\usepackage{listings}
\usepackage{eurosym}
\usepackage{import}
\usepackage{multirow}
\usepackage{tabularx}


%------------------------ Plot packages ----------------------------
\usepackage{tikz}
\usetikzlibrary{positioning,arrows,calc}
\usepackage{pgfplots}
\usepackage{pgf}
\usepackage{units}
\usepackage{metalogo}
\usepackage{graphicx}
\usepackage{caption}
\usepackage{subcaption}
\usepackage[mode=buildnew]{standalone}% requires -shell-escape

% Finally the hyperref package is used for pdf files.
% This can be commented out for printed versions.
\usepackage[pdfusetitle,colorlinks,plainpages=false,hidelinks]{hyperref}


\begin{document}


\begin{preface}
  I would like to thank everybody who kept me busy the last year,
  especially my promotor and my assistants. I would also like to thank the
  jury for reading the text. My sincere gratitude also goes to my family
  for supporting me trough my studies.
\end{preface}

\tableofcontents*

\begin{abstract}
  Speech recognition is concerned with transcribing what is said in a
  recoding of spoken language. In machine learning terms this process is
  called sequence labeling. A recoding consists of a chain of frames,
  this chain can be split up into several sequences, these make
  up words or phonemes, which must be labeled. The sequence of labels forms
  the transcription. \\
  The meaning of speech depends on context, therefore a good system needs to
  take it into account. Classical feed-forward networks fail to do that, which
  is why this system will mainly consist of recurrently connected Long Short Term
  Memory (LSTM) blocks. Inspired by the recurrent connects of neurons in the human
  brain, LSTM-RNNs have the ability to store information over long time periods,
  an important requirement in take context into account. \\
  In order to train machine learning systems, speech and transcription text
  pairs are used. The text contains the exact information of what is said in
  the recording, but where in the recoding which word or sound is said is unknown.
  In other words text to speech alignment is missing. Aligning the data will be an
  important issue in this thesis.    

\end{abstract}

% A list of figures and tables is optional
%\listoffigures
%\listoftables
% If you only have a few figures and tables you can use the following instead
\listoffiguresandtables
% The list of symbols is also optional.
% This list must be created manually, e.g., as follows:
\chapter{List of Abbreviations and Symbols}
\section*{Abbreviations}
\begin{flushleft}
  \renewcommand{\arraystretch}{1.1}
  \begin{tabularx}{\textwidth}{@{}p{12mm}X@{}}
    ConvNet   & Convolutional neural network \\
    MSE   & Mean Square error \\
    PSNR  & Peak Signal-to-Noise ratio \\
  \end{tabularx}
\end{flushleft}
\section*{Symbols}
\begin{flushleft}
  \renewcommand{\arraystretch}{1.1}
  \begin{tabularx}{\textwidth}{@{}p{12mm}X@{}}
    42    & ``The Answer to the Ultimate Question of Life, the Universe,
            and Everything'' according to \cite{h2g2} \\
    $c$   & Speed of light \\
    $E$   & Energy \\
    $m$   & Mass \\
    $\pi$ & The number pi \\
  \end{tabularx}
\end{flushleft}

% Now comes the main text
\mainmatter

\import{../chapters/}{literature}
\import{../chapters/}{ctcExperiments}
\import{../chapters/}{lasExperiments}
% ... and so on until
%\import{../chapters/}{chap-n}
%\import{../chapters/}{conclusion}

% If you have appendices:
%\appendixpage*          % if wanted
%\appendix
%\chapter{The First Appendix}
\label{app:A}
Appendices hold useful data which is not essential to understand the work
done in the master thesis. An example is a (program) source.
An appendix can also have sections as well as figures and references\cite{h2g2}.

\section{More Lorem}
\lipsum[50]

\subsection{Lorem 15--17}
\lipsum[15-17]

\subsection{Lorem 18--19}
\lipsum[18-19]

\section{Lorem 51}
\lipsum[51]

%%% Local Variables: 
%%% mode: latex
%%% TeX-master: "thesis"
%%% End: 

% ... and so on until
%\chapter{The Last Appendix}
\label{app:n}
Appendices are numbered with letters, but the sections and subsections use
arabic numerals, as can be seen below.

\section{Lorem 20-24}
\lipsum[20-24]

\section{Lorem 25-27}
\lipsum[25-27]

%%% Local Variables: 
%%% mode: latex
%%% TeX-master: "thesis"
%%% End: 


\backmatter
% The bibliography comes after the appendices.
% You can replace the standard "abbrv" bibliography style by another one.
\bibliographystyle{abbrv}
\bibliography{../chapters/references}

\end{document}

%%% Local Variables: 
%%% mode: latex
%%% TeX-master: t
%%% End: 
