\documentclass{beamer}
%
% Choose how your presentation looks.
%
% For more themes, color themes and font themes, see:
% http://deic.uab.es/~iblanes/beamer_gallery/index_by_theme.html
%
\mode<presentation>
{
  \usetheme{Darmstadt}      % or try Darmstadt, Madrid, Warsaw, ...
  \usecolortheme{crane} % or try albatross, beaver, crane, ...
  \usefonttheme{default}  % or try serif, structurebold, ...
  \setbeamertemplate{navigation symbols}{}
  \setbeamertemplate{caption}[numbered]
  \usepackage{tikz}
  \usepackage{pgfplots}
  \usepackage{pgf}
  \usepackage{units}
  \usepackage{metalogo}
  \usepackage{graphicx}
  \usepackage{caption}
  \usepackage{subcaption}
  \usepackage[mode=buildnew]{standalone}% requires -shell-escape
  \usepgfplotslibrary{groupplots}

} 

\usepackage[english]{babel}
\usepackage[utf8x]{inputenc}
\setbeamertemplate{footline}[frame number]

\title{Thesis intermediate Presentation}
\author{Moritz Wolter}
\date{\today}

\begin{document}

\begin{frame}
  \titlepage
\end{frame}


% Uncomment these lines for an automatically generated outline.
%\begin{frame}{Outline}
%  \tableofcontents
%\end{frame}


\section{Deep Learning}
\begin{frame}{Long Short Term Memory (LSTM)}
\begin{figure}
\includestandalone[height=6.5 cm]{../tikz/lstm}
\caption{Visualization of the LSTM architecture}
\label{fig:lstm}
\end{figure}
\end{frame}

\begin{frame}{Bidirectional BLSTM}




\end{frame}


\section{Listen Attend and Spell}
\begin{frame}{The LAS-Architecture}
\begin{figure}
\includestandalone[height=6.5 cm]{../tikz/lasArcBottomUp}
\caption{The LAS architecture}
\label{fig:las}
\end{figure}
\end{frame}


\end{document}
