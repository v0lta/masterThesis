\chapter{Methodology}
\label{cha:problemStatement}


\section{Pylint}
When writing code it is important to ensure that not only the author, but also future readers and coworkers are able to understand it. An important part of ensuring future usability is to document functionality. However readable code is not just well documented, but also respects well established coding conventions. An important collection of python code conventions is the python foundations PEP 8 style guide \cite{VanRossum2001}. However following these conventions is tedious and easily forgotten. Additionally software development is team work and the best way to ruin the esprit de corps is an environment where team members constantly complain to each other about code convention violations.
A good way to enforce decent coding style is to get the computer to help. In the python world a useful tool for this purpose is \texttt{pylint}\footnote{https://pylint.readthedocs.io/en/latest/intro.html}. \texttt{Pylint} is a code analysis tool that looks for patterns in the code that might indicate bad style, for example incorrect variable names, missing documentation, wrong indentation and so forth. 
Being an interpreted language python code does not need to be compiled. This advantage comes with an important drawback, because during compiling many errors in the code are found and reported to the programmer. Python can do the same thing during runtime, but it crashes when an error is encountered and already computed data is often lost during those crashes. Pylint therefore taps into the python interpreter to scan the code for errors like a compiler. By alerting the programmer to potential issues in the code before execution, pylint can decrease runtime crashes and increase development efficiency.    