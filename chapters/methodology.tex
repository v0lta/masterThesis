\chapter{Methodology}
\label{cha:methods}
This chapter lays out important software development concepts and methods used to develop the code base, on which this project rests.

\section{Object oriented programming}
In computer science object oriented programming is an important paradigm that uses objects to group related data and code into object fields and routines. In addition to better grouping objects allow to encapsulate data, shielding it from access. Good objects will not only encapsulate data but at the same time provide the tools necessary to manipulate what is stored inside. Ideally future users or other team members will be able to use the object without in depth knowledge of it's internals, which makes team software development easier.  
An important goal of this thesis project is code re-usability, additionally the work developed should fit nicely into the existing speech recognition toolbox. To allow future users to switch between various algorithms with ease an object oriented design philosophy is followed. This approach enables future users to switch from a CTC based to a listen attend and spell based recognition system by replacing a CTC graph object with a listen attend and spell graph object.

\section{Shape invariants}
Writing code implementing artificial neural networks is difficult, because the complete project is not likely to work unless all issues in it's code have been fixed. A single bug can prevent experiments from working, in such a case the only feedback the programmer gets is a negative experimental result. In order to prevent incorrect implementations of recurrent neural networks it is important to strictly enforce shape invariants of state tensors in loops. This means that the shape of the tensors going into a loop body must have the same size when they leave. This way one can be certain, that for example the network does not change the number of label probabilities it predicts over time.
However sometimes the shape of a tensor must be altered, for example if newly computed results become available and a tensor's dimension must be increased to create storage space. 
In such cases shape invariants must be explicitly turned off, that way data can be accumulated over several loop iterations. Shape invariants force programmers to consciously specify the tensors and the corresponding dimensions where data accumulate. Thus making these important parts of the code explicitly visible, while at the same time preventing unwanted data accumulation and memory consumption elsewhere.  


\section{Code quality}
During coding it is important to ensure that not only the author, but also future readers and coworkers are able to understand the source. An important part of ensuring future usability is to document functionality. However readable code is not just well documented, but also respects well established coding conventions. An important collection of python code conventions is the python software foundation's PEP 8 style guide \cite{VanRossum2001}. However following these conventions is tedious and easily forgotten. Additionally software development is team work and the best way to ruin the esprit de corps is an environment where team members constantly complain to each other about code convention violations.
A good way to enforce decent coding style is to get computers to help. In the python world a useful tool for this purpose is \texttt{pylint}\footnote{https://pylint.readthedocs.io/en/latest/intro.html}. \texttt{Pylint} is a code analysis tool that looks for code patterns, that might indicate bad style, for example incorrect variable names, missing documentation, wrong indentation and so forth. 
Being an interpreted language python code does not need to be compiled. This advantage comes with an important drawback, because during compiling many errors in the code are found and reported to the programmer. Python can do the same thing during runtime, but it crashes when an error is encountered and already computed data is often lost during those crashes. \texttt{Pylint} taps into the python interpreter to scan the source for errors like a compiler. By alerting the programmer to potential issues in the code before execution, \texttt{pylint} can decrease the number of runtime crashes and increase development efficiency.    

\section{Version control}
Version control software tracks changes to the codebase under development and allows authors to back up versions of the code as revisions. Git\footnote{\url{https://git-scm.com/}} is a popular version control tool. Git keeps all past revisions in a graph. The full graph is a recording of the project's history since its inception. However version control is more than simply keeping a record of past changes. Depending on the project adding new features bears the risk of breaking older code. If things go badly the code can end up in a state, where neither any of the old nor the desired new functionalities work. In order to always maintain a working core, version control software should be used to split of the development of new functionalities into branches. Should serious problems arise, which where not anticipated when the a feature was planned, the corresponding new branch can simply be discarded, leaving the core code intact. Should newly written functionality work as intended, it's branch is merged into the core code, which remained fully functional at all times.
Additionally working with branches facilitates software development in a Team, as different members can work independently on their own branches. Without branching working with several people on the same files can become tedious, if conflicting changes are made. Branched work lets team members focus on development first, potential conflicts are then resolved later, when the various tree branches are merged.